\documentclass[12pt, titlepage, twoside]{amsart}

\usepackage[a4paper, margin=1in]{geometry}
\usepackage{amsmath}
\usepackage[foot]{amsaddr}
\usepackage{amssymb}
\usepackage{amsthm}
\usepackage{enumitem}
\usepackage[dvipsnames]{xcolor}
\usepackage{parskip}
\usepackage{graphicx}
\usepackage{tikz}
\usepackage{listings}
\usepackage{lipsum}
\usepackage[cache=false]{minted}
\usepackage{hyperref}

\newcommand{\R}{\ensuremath{\mathbb R}}
\newcommand{\Z}{\ensuremath{\mathbb Z}}
\newcommand{\N}{\ensuremath{\mathbb N}}
\newcommand{\C}{\ensuremath{\mathbb C}}

\hypersetup{
  colorlinks=true,
  linkcolor=Orchid,
  urlcolor=ProcessBlue
}

\usemintedstyle{stata-dark}

\raggedright

\begin{document}

\title[ScottyRank.jl]{ScottyRank.jl: An Implementation of PageRank \& HITS}

\author{Siyuan Chen}
\author{Michael Zhou}
\email{siyuanc2@andrew.cmu.edu}
\email{mhzhou@andrew.cmu.edu}
\date{November 2021}

\maketitle

\section{Mathematical Background}

\subsection{Definitions}

Positive matrices are defined as matrices with positive entries.

Square matrices are defined as matrices with the same number of columns and rows.

Markov matrices are defined as square matrices with nonnegatives entries and column sum $1$ across all of its columns.
Note that for a $n\times n$ matrix $M$, the latter condition is equivalent to $M^T\vec{1} = \vec{1}$,
where $\vec{1}\in\R^n$ has all ones as components.

Positive Markov matrices are defined as, well, positive Markov matrices.

\subsection{Properties}


(Perron-Frobenius Theorem)
Let $A$ be a positive square matrix.
Then $A$ has an eigenvalue $\lambda_0$ such that all other 








\end{document}
