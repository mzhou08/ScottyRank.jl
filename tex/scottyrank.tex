\documentclass[12pt, titlepage, twoside]{amsart}

\usepackage[a4paper, margin=1in]{geometry}
\usepackage{amsmath}
\usepackage[foot]{amsaddr}
\usepackage{amssymb}
\usepackage{amsthm}
\usepackage{enumitem}
\usepackage[dvipsnames]{xcolor}
\usepackage{hyperref}
\usepackage{parskip}
\usepackage{graphicx}
\usepackage{tikz}
\usepackage[cmintegrals, cmbraces]{newtxmath}
\usepackage{ebgaramond-maths}
\usepackage[T1]{fontenc}
\usepackage{listings}
\usepackage{lipsum}
% \usepackage{microtype}
% WILL TIDY UP PACKAGES LATER

\newcommand{\R}{\ensuremath{\mathbb R}}
\newcommand{\Z}{\ensuremath{\mathbb Z}}
\newcommand{\N}{\ensuremath{\mathbb N}}
\newcommand{\F}{\ensuremath{\mathbb F}}
\newcommand{\C}{\ensuremath{\mathbb C}}

\renewcommand{\vec}[1]{\ensuremath{\mathbf{#1}}}
\newcommand{\norm}[1]{\ensuremath{\lVert #1\rVert}}
\newcommand{\std}[1]{\ensuremath{\frac{#1}{\norm{#1}}}}
\newcommand{\proj}[2]{\ensuremath{\mathrm{proj}_{#1}{#2}}}

\theoremstyle{remark}
  \newtheorem*{cl}{Claim}
  \newtheorem*{pf}{Proof}

\setenumerate{label=(\alph*)}

\hypersetup{
  colorlinks=true,
  linkcolor=Orchid,
  urlcolor=ProcessBlue
}

\begin{document}

\title[ScottyRank.jl]{ScottyRank.jl: A Julia Implementation of PageRank and HITS}

\author{Siyuan Chen}
\author{Michael Zhou}
\email{siyuanc2@andrew.cmu.edu}
\email{mhzhou@andrew.cmu.edu}
\date{November 2021}

\begin{abstract}
PageRank is a method famously used by Google to determine the relative importance of different websites. More generally, PageRank can be applied to any directed graph of objects, where one wishes to find the most "important" nodes, as determined by a combination of the number of nodes pointing to it and the number of nodes that it points to. The HITS variation of the PageRank algorithm adds on "hub" and "authority" scores, which further distinguishes between nodes pointing to many others and nodes with many others pointing to it, respectively.

In our project, we implemented both the PageRank and HITS algorithms using Julia to better understand the linear algebra insights behind the two algorithms.
\end{abstract}

\maketitle

\tableofcontents

\section{Introduction}

\lipsum[1]

\section{Background}

\lipsum[1]

\begin{lstlisting}
export Vertex, Graph

struct Vertex
  index::UInt32
  in_neighbors::Vector{UInt32}
  out_neighbors::Vector{UInt32}
end

struct Graph
  num_vertices::UInt32
  vertices::Vector{Vertex} # sorted by index
end
\end{lstlisting}

\lipsum[1]

\section{Method}

\lipsum[1]

\section{Results}

\lipsum[1]

\section{Discussion}

\lipsum[1]

\section{Conclusion}

\lipsum[1]

\end{document}
